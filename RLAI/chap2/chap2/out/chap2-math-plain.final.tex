% MathMode: plain, MathRender: svg, MathDpi: 300, MathEmbedLimit: 524288, MathScale: 105, MathBaseline: 0, MathDocClass: [10pt]book, MathImgDir: math, MathLatex: latex, MathSvgFontFormat: "", MathSvgSharePaths: True, MathSvgPrecision: 3, Dvisvg: dvisvgm
\documentclass[10pt]{book}
% generated by Madoko, version 1.1.2
%mdk-data-line={1}
\newcommand\mdmathmode{plain}
\newcommand\mdmathrender{svg}
\usepackage[heading-base={2},section-num={false},bib-label={true},fontspec={true}]{madoko2}
%mdk-data-line={1;/usr/local/lib/node_modules/madoko/lib/../styles/presentation.mdk:79}

    \ifbeamer\relax\else
      \providecommand{\usetheme}[2][]{}
      \providecommand{\usecolortheme}[2][]{}
      \providecommand{\usefonttheme}[2][]{}
      \providecommand{\pause}[1][]{}
      \providecommand{\AtBeginSection}[2][]{}
      \providecommand{\AtBeginSubsection}[2][]{}
      \providecommand{\AtBeginSubsubsection}[2][]{}
      \providecommand{\AtBeginPart}[2][]{}
      \providecommand{\AtBeginLecture}[2][]{}
      \providecommand{\theoremstyle}[2][]{}
      \makeatletter
      \def\newtheorem{\@ifstar\newtheoremx\newtheoremx}
      \makeatother
      \providecommand{\newtheoremx}[3][]{}{}
    \fi

\begin{document}


\begin{mdSnippets}
%mdk-data-line={10}
%mdk-data-line={50}
\begin{mdInlineSnippet}[e358efa489f58062f10dd7316b65649e]%mdk
$t$\end{mdInlineSnippet}%mdk
%mdk-data-line={52}
\begin{mdInlineSnippet}[7b8b965ad4bca0e41ab51de7b31363a1]%mdk
$n$\end{mdInlineSnippet}%mdk
%mdk-data-line={54}
\begin{mdInlineSnippet}[3d1dfe70cdc0d574aa6cf3e228a57166]%mdk
$r_t$\end{mdInlineSnippet}%mdk
%mdk-data-line={98}
\begin{mdInlineSnippet}[a42da9c575b4e6bcbde63bd2e43f6ce9]%mdk
$e = mc^2$\end{mdInlineSnippet}%mdk
%mdk-data-line={101}
\begin{mdDisplaySnippet}[5094bc6c72a97e9f9f69ae85f34ed065]%mdk
\[%mdk-data-line={102}
\int_{-\infty}^\infty e^{-a x^2} d x = \sqrt{\frac{\pi}{a}} 
\]%mdk
\end{mdDisplaySnippet}%mdk
%mdk-data-line={264}

\end{mdSnippets}

\end{document}
